%!TEX root =../main.tex 
\setcounter{chapter}{-1} % <-- Work around so that the template chapter gets number 0.
\chapter{The ComSys Template}
\label{chap:template}

This chapter explains \textbf{how} to write a thesis in \LaTeX.
The following chapters give the base structure to use.
Also, they give some hints on \textbf{what} to write in a thesis.

\section{General Rules}
\label{sec:general_rules}
%==============================================================================
This \gls{ComSys} \LaTeX template \textbf{must} be used for all theses (Bachelor and Master) of the \gls{ComSys} work group.
Additionally, the following rules have to be followed:
\begin{enumerate}
	\item Without permission of your supervisor, do not add, remove, reorder or rename the chapter titles (\texttt{\textbackslash{}chapter} command)!
	\item Please extend, rename and improve the sections (\texttt{\textbackslash{}section} command), subsections (\texttt{\textbackslash{}subsection} command) and subsubsections.
	\item Use \texttt{make} to generate the PDF
	\begin{itemize}
		\item You can use the PCs in the MIoT-Lab (building 29, room 321), which have all required tools installed
		\item You can use your own PC, but you should have to use a POSIX-compliant OS (Linux, Mac OS (X), FreeBSD, ...) with GNU Make and \texttt{texlive-full} installed.
		You may need to install \texttt{latexmk}~\cite{latexmk} as well.
		For building the glossary a local or global \texttt{.latexmkrc} file is needed, see \href{https://www.ctan.org/texarchive/support/latexmk/example_rcfiles}{glossary\_latexmkrc}.
	\end{itemize}
	\item \textit{\textbf{\textsc{Never ever ever commit temporary files!}}}
	\begin{itemize}
		\item All files with the extensions \texttt{.aux}, \texttt{.log}, \texttt{.synctex}, \texttt{.synctex.gz}, \texttt{.toc}, \texttt{.bbl}, etc. and \textbf{the generated PDF} can be re-build from source and \textbf{must not} be commited to the repositry
		\item Running \texttt{make clean} will clear all temporary files
		\item You can use the \texttt{global-ignores} feature in the subversion configuration of your machine to prevent temporary files from uploading
		\item Run \texttt{svn status} before running \texttt{svn commit} to check if no temporary files is going to be commited
	\end{itemize}
	\item Check using \texttt{svn status} if all sources files are added to the repository before commiting
	\item Check that \texttt{make} is running without errors before commiting
	\item Commit often: This fights data losses and allows you to revert changes if needed
	\item Use \texttt{svn mv} instead of \texttt{mv}, \texttt{svn cp} instead of \texttt{cp} and \texttt{svn rm} instead of \texttt{rm} within the Repository
	\item Prefer vector graphics (\texttt{.svg}, \texttt{.eps}, \texttt{.pdf}) over \texttt{.png}, \texttt{.bmp}, \texttt{.jpg}
	\begin{itemize}
		\item \LaTeX cannot directly use \texttt{.svg} vector graphics, but you can convert them to vectorized \texttt{.pdf} files using \href{http://www.inkscape.org/}{Inkscape}
		\item You can directly create vectorized graphics in \LaTeX using TikZ, e.g. see \href{http://www.texample.net/tikz/}{these examples}
	\end{itemize}
    \item Use a seperate line for each sentence in your \texttt{.tex} files (for version control)
	\item Add papers, websites and other sources to \texttt{bib/bibliography.bib}
	\begin{itemize}
		\item Instead of working with the source, you can use \href{http://jabref.sourceforge.net/}{JabRef} to graphically edit these files
		\item Most papers have a \gls{DOI}, which enabled you to use \href{https://doi2bib.org/}{doi2bib.org} to look up a BibTeX entry
		\item Most journals and conferences allow you to download a BibTeX entry free of charge
		\item For RFCs BibTeX entries are also provided online
	\end{itemize}
	\item Style of floats (\texttt{figure}, \texttt{table}, \ldots)
	\begin{itemize}
		\item They are in general best placed on top of a page (\texttt{[t]})
		\item Huge floats are best placed as a single page (\texttt{[p]})
		\item Try to use \texttt{[width=\textbackslash{}linewidth]} whenever possible
		\item \textbf{Always} provide a \texttt{caption} describing short and precisely \textbf{what} the figure/table is showing
		\item Add a legend if further explanation is needed on \texttt{how} this is shown
	\end{itemize}
	\item Use \texttt{lstinputlisting} to include source code from an external file (or the \texttt{lstlisting} environment for short listings)
	\item You need to build the glossary and bibliography manually if you do not use the provided Makefile.
	The command \texttt{makeindex} is used for generating the glossary, after you built your latex document once.
	The workflow is the following:
	\begin{itemize}
	 \item build your latex document
	 \item build your list of glossaries 
        \begin{verbatim}
        makeindex -s main.ist -t main.glg -o main.gls main.glo
        \end{verbatim}
        build your list of acronyms
        \begin{verbatim}
        makeindex -s main.ist -t main.alg -o main.acr main.acn
        \end{verbatim}
    \item build your bibliography
    \item build your latex document (x2)
	\end{itemize}
\end{enumerate}

\section{\LaTeX Examples}

This template includes examples for the following items:

\begin{itemize}
    \item Enumerations, like used here
	\item Numbered lists in Section~\ref{subsec:numbered_lists}
	\item Tables in Section~\ref{subsec:tables}
	\item Figures in Section~\ref{subsec:figures}
    \item Byte fields in Section~\ref{subsec:byte_fields}
	\item Math mode with over multiple lines in Section~\ref{subsec:math_mode_multiple}
	\item Source code in Section~\ref{sec:sourcecode}
	\item Citing a literature entry on Page~\pageref{cha:introduction}
	\item Acronyms and glossary, used for instance in Chapter~\ref{cha:introduction} and defined in \texttt{text/glossary.tex}.
	For more information please take a look at the package \texttt{glossaries}.
	\item References to sections, figures, source code, tables etc. can be found throughout this document
	\item Math mode within a line: $\exists x \in \{1,\frac{3}{2},2,\ldots,9\}$
	\item An annotation with the Todo-package \todo{this is an example of todo} is shown here
\end{itemize}

\subsection{Numbered Lists}
\label{subsec:numbered_lists}

\begin{enumerate}
	\item This is the first item
	\item This is a second
	\item This one has sub-items:
	\begin{enumerate}
		\item Subitem 1
		\item Subitem 2
	\end{enumerate}
	\item This is the last item
\end{enumerate}

\subsection{Tables}
\label{subsec:tables}

\begin{table}[t]
 \begin{tabular}{L{0.25}*{3}{R{0.25}}}
  \toprule
  First Column & Second Column & Third Column & Fourth Column \\
  \midrule
  first line & some content & ... & ... \\
  second line & other content & ... & ... \\
  \bottomrule
  \end{tabular}
  \caption{Example Table}
  \label{table:example}
\end{table}

An example table is shown in Table~\ref{table:example}.
Take care that the column sizes add up to \texttt{1}.


\subsection{Figures}
\label{subsec:figures}

\begin{figure}
 \centering
 \includegraphics[width=0.2\linewidth]{graphics/smiley}
 \caption{Example Figure}
 \label{fig:example}
\end{figure}

An example figure is shown in Figure~\ref{fig:example}.


\subsection{Byte Fields}
\label{subsec:byte_fields}

\begin{figure}[t]
\centering
\begin{bytefield}{32}
\bitheader{0, 15, 31} \\
\begin{leftwordgroup}{Header}
\bitbox{16}{Source} & \bitbox{16}{Destination} \\
\bitbox{16}{Length} & \bitbox{16}{Checksum} 
\end{leftwordgroup} \\
\begin{leftwordgroup}{Payload}
\wordbox[lrt]{1}{Data} \\
\skippedwords \\
\wordbox[lrb]{1}{}
\end{leftwordgroup} \\
\end{bytefield}
\caption{Example Byte Field}
\label{fig:bytefield:example}
\end{figure}

An example byte field is shown in Figure~\ref{fig:bytefield:example}.

\subsection{Math Mode Over Multiple Lines}
\label{subsec:math_mode_multiple}
In the following, an example for consecutive equations without numbering is shown.
For equations with numbering use the \texttt{align} environment.

\begin{align*}
0       &= ax^2 + bx^2 + c & (a \neq 0) \\
x_{1,2} &= \frac {-b \pm \sqrt{b^2 - 4ac}}{2a} \\
\end{align*}

\newpage
\section{Further Documentation}
We collected further documents on the work with Latex on our homepage, which can be found 
\href{http://www.comsys.ovgu.de/THESIS/Technical+Writing+and+Presentation.html}{\texttt{here}}.
In general, Latex packages an their documentation can be found on \href{http://www.ctan.org/}{Comprehensive TeX Archive Network}.


Links to the most important packages:
\begin{itemize}
	\item \href{http://tug.ctan.org/cgi-bin/ctanPackageInformation.py?id=bytefield}{bytefield - Create illustrations for network protocol specifications}
	\item \href{http://tug.ctan.org/cgi-bin/ctanPackageInformation.py?id=colortbl}{colortbl - Add colour to LaTeX tables}
	\item \href{http://tug.ctan.org/cgi-bin/ctanPackageInformation.py?id=eqnarray}{eqnarray - More generalised equation arrays with numbering}
	\item \href{http://tug.ctan.org/cgi-bin/ctanPackageInformation.py?id=glossaries}{glossaries - Create glossaries and lists of acronyms}
	\item \href{http://tug.ctan.org/cgi-bin/ctanPackageInformation.py?id=graphicx}{graphicx - Enhanced support for graphics}
	\item \href{http://tug.ctan.org/cgi-bin/ctanPackageInformation.py?id=listings}{listings - Typeset source code listings using LaTeX}
	\item \href{http://tug.ctan.org/cgi-bin/ctanPackageInformation.py?id=pdflscape}{pdflscape - Make landscape pages display as landscape}
	\item \href{http://tug.ctan.org/cgi-bin/ctanPackageInformation.py?id=supertabular}{supertabular - A multi-page tables package}
	\item \href{http://tug.ctan.org/cgi-bin/ctanPackageInformation.py?id=xcolor}{xcolor - Driver-independent color extensions for LaTeX and pdfLaTeX}
\end{itemize}
