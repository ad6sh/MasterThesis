\section{Experiments}
\label{sec:experiments}
%==============================================================================
Describe the experiments you conducted.
Detail the steps required to reproduce your data.
Explain in detail what tools you used, what data was collected etc.

Make sure to design your experiments so that they allow drawing conclusions from.
E.g. if you want to evaluate how changes in property $x$ affect $y$, you want to perform the experiments for different values of $x$.
Every other property of the experimental setup should be identical.
Otherwise you will later on not be ably to tell if the different observations for $y$ are due to the changes in $x$ or due to the other things changed in the experiment setup.

\textbf{Repeat your experiments!}
If you have less than 30 data points for one experiment setup, your observations might be due to change and not reproducible.

\textbf{Try to sketch the data you would expect from your experiments in a mock-up diagram.}
How would the observed data look like if your contribution ``works like a charm''.
How would it look like if your contribution ``failed bitterly''?
If the observed data would only differ little or not at all, you will have to redesign your experiment.

\textbf{Do not start to evaluate or interpret the experiments here.
This goes to the evaluation in Section~\ref{sec:evaluation}!}
