\chapter{Introduction}
\label{cha:introduction}
%==============================================================================
Introduce the topic of your thesis here.
Often it is a good idea to start more general and then narrow it down to your specific topic.


\section{Motivation}
\label{sec:motivation}
%==============================================================================
Describe why the topic for your thesis is worth to investigate.
E.g. if your topic is related to the \gls{IoT} you could present some predictions of how many \gls{IoT} nodes will be deployed in the near future, or how big the \gls{IoT} market will be.
Make sure to properly cite, e.g. according to IHS~\cite{ihs:2016:iotdevices} the number of \gls{IoT} devices worldwide will reach \num{75.44} billions by 2025.

\section{Thesis Structure}
\label{sec:structure}
%==============================================================================
Outline the structure of your thesis.
Keep in mind to write ``the next chapter'' (lower case), but in Chapter~\ref{cha:related_work} (upper case).
Similar section, figure, and table will become upper case when referring to a specific section/figure/\ldots by number.
Use the non-breaking space (tilde) between Chapter/Section/Figure/Table and the number.
Use the \texttt{\textbackslash{}ref} LaTeX command for references, so that LaTeX will insert the correct numbers for you.
